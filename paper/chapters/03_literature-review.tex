\chapter{Literature Review}
\label{chap:lr}

%%%%%%%%%%%%%%%%%%%%%%%%%%%%%%%%%%%%%%%%%%%%%%%%%%%%%%%%%%%%%%%%%%%%%%%%%%%%%%%%
% Identify groups of related work (e.g. variations of an algorithm, implementations, theoretical results, etc.)
% This section may include papers (published research, pre-prints), existing libraries, software, blog posts, etc.
%%%%%%%%%%%%%%%%%%%%%%%%%%%%%%%%%%%%%%%%%%%%%%%%%%%%%%%%%%%%%%%%%%%%%%%%%%%%%%%%

\section{Typechecking Algorithms}

In 1928 Damas and Milner \cite{DamasMilner1982_TypeSchemes} have shown an implementation of \texttt{let}-generalization — Algorithm~$\mathcal{W}$, which, however, is inefficient as it requires looking through the entire type environment.
A more efficient algorithm has been discovered by R\'emy~\cite{Remy1992_SortedEqTheoryTypes} and shown to be successfully adopted in the OCaml programming language \cite{Kiselyov2022_OCamplTypeChecker}, although it is not widely-known.

\section{Expression Problem}

In 1998 Wadler formulated an Expression Problem \cite{ExpressionProblem} that involves extending a program with new data types and new operations without modifying existing code or losing type safety.

Sweirstra's proposed a solution with data types à la carte \cite{Swierstra2008_a_la_carte} to the problem, and such approach was adopted in Free Foil.

\section{Similar Works}

\texttt{hypertypes} \cite{hypertypes} is another Haskell library that solves the expression problem and provides a generic implementation of Hindley-Milner type inference based on work of Rémy \cite{Remy1992_SortedEqTheoryTypes}.

Another work by Heeren \textit{et al.} \cite{Heeren2002GeneralizingHindleyMilner} generalized $\mathcal{W}$ and $\mathcal{M}$ Hindley-Milner type inference algorithms. However, the focus of their work is to modify the original (inefficient \cite{Remy1992_SortedEqTheoryTypes}) algorithms in order to improve error messages generation but not to make development of a new typechecker easier.
