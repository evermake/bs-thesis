\documentclass[oneside,final,14pt,a4paper]{extreport}

\usepackage{tempora} % Times New Roman alike font  

% Use geometry package as vmargin conflicts with pdfpages.
\usepackage[
  a4paper,
  includeheadfoot,
  left=2.5cm,
  right=2cm,
  top=2cm,
  bottom=2cm,
  headsep=10mm,
  footskip=13mm
]{geometry}
 
\usepackage{setspace}
\setstretch{1.5}
\usepackage{indentfirst}
\parindent=1.25cm

%%%%% ADDED TO SUPPORT TT BOLD FACES %%%%
\DeclareFontShape{OT1}{cmtt}{bx}{n}{<5><6><7><8><9><10><10.95><12><14.4><17.28><20.74><24.88>cmttb10}{}
\renewcommand{\ttdefault}{cmtt}
%%%%% END %%%%%%%%%%%%%%%%%%%%%%%%%%%%%%% 

\usepackage{atbegshi,picture}
\usepackage[T1,T2A]{fontenc}
\usepackage[utf8]{inputenc}

\usepackage[english]{babel}
\usepackage[backend=biber,style=ieee,autocite=inline]{biblatex}
\bibliography{../ref.bib}
\DefineBibliographyStrings{english}{%
  bibliography = {References},}
\usepackage{blindtext}

\usepackage{pdfpages}
\newenvironment{bottompar}{\par\vspace*{\fill}}{\clearpage}
\usepackage{amsmath,amsfonts}
\usepackage{url}
\usepackage{amsthm}
\newtheorem{theorem}{Theorem}
\newtheorem{corollary}{Corollary}
\newtheorem{lemma}{Lemma}
\newtheorem{proposition}{Proposition}
\theoremstyle{definition}
\newtheorem{definition}{Definition}
\theoremstyle{remark}
\newtheorem*{remark}{Remark}
\theoremstyle{remark}
\newtheorem*{example}{Example}

\usepackage{float}
\usepackage{graphicx}
\graphicspath{{figs/}} %path to images
\usepackage{array}
\usepackage{multirow,array}
\usepackage{caption}
\usepackage{subcaption}
\usepackage{hyperref}
\hypersetup{colorlinks=true, linkcolor=black, citecolor=black}
\usepackage{paralist}
\usepackage{listings}
\usepackage{minted} % Added for code highlighting
\usepackage{zed-csp}
\usepackage{fancyhdr}
\usepackage{csquotes}
\usepackage{color}
% \usepackage{anyfontsize}
% \usepackage{mathptmx}
% \usepackage{t1enc}

\definecolor{codegreen}{rgb}{0,0.6,0}
\definecolor{codegray}{rgb}{0.5,0.5,0.5}
\definecolor{codepurple}{rgb}{0.58,0,0.82}
\definecolor{backcolour}{rgb}{0.95,0.95,0.92}

\lstdefinestyle{mystyle}{
  backgroundcolor=\color{backcolour},
  commentstyle=\color{codegreen},
  keywordstyle=\color{magenta},
  numberstyle=\tiny\color{codegray},
  stringstyle=\color{codepurple},
  basicstyle=\ttfamily\footnotesize,
  breakatwhitespace=false,
  breaklines=true,
  captionpos=b,
  keepspaces=true,
  numbers=left,
  numbersep=5pt,
  showspaces=false,
  showstringspaces=false,
  showtabs=false,
  tabsize=2
}
\lstset{style=mystyle}

\usepackage{mathpartir}
\usepackage{chngcntr}
\usepackage{upgreek} 
\usepackage{bm}
\usepackage{hyperref}
\usepackage{booktabs}
\usepackage{multirow}
\usepackage{longtable}
\usepackage{ebproof}
\usepackage[font=singlespacing, labelfont=bf]{caption}
%Hints
\newcommand\pic[1]{(Fig. \ref{#1})} %Ref on figure
\newcommand\tab[1]{(Tab. \ref{#1})} %Ref on table

\setlength{\headheight}{32.0976pt}
\usepackage{enumitem}
\newlist{inlinelist}{enumerate*}{1}
\setlist*[inlinelist,1]{%
  label=(\arabic*),
}

% \setcounter{secnumdepth}{4}
\captionsetup[table]{labelfont={normalfont}, name={TABLE}, labelsep={newline}}
\setlength{\parindent}{2em} 
\DeclareCaptionLabelSeparator{figSep}{.\quad}
\captionsetup[figure]{labelfont={normalfont}, name={Fig.}, labelsep=period}
\counterwithin{figure}{chapter}

% \usepackage{titlesec}
% \titleformat{\section}[hang]{\fontsize{20}{24}\selectfont\filcenter}{\Roman{section}}{1em}{}
% \titleformat{\subsection}[hang]{\itshape}{\Alph{subsection}.}{1em}{}[]
% \titleformat{\subsubsection}[runin]{\itshape}{\arabic{subsubsection})}{1em}{}[$:$]
% \titlespacing{\subsubsection}{1em}{1em}{1em}
% \titleformat{\paragraph}[runin]{\itshape}{\alph{paragraph})}{1em}{}[$:$\quad]
% \titlespacing{\paragraph}{2em}{1em}{1em}

\usepackage{placeins} % for \FloatBarrier

\pagestyle{fancyplain}

% remember section title
\renewcommand{\chaptermark}[1]%
	{\markboth{\chaptername~\thechapter~--~#1}{}}

% subsection number and title
\renewcommand{\sectionmark}[1]%
	{\markright{\thesection\ #1}}

\rhead[\fancyplain{}{\bf\leftmark}]%
      {\fancyplain{}{\bf\thepage}}
\lhead[\fancyplain{}{\bf\thepage}]%
      {\fancyplain{}{\bf\rightmark}}
\cfoot{} %bfseries


\newcommand{\dedication}[1]
   {\thispagestyle{empty}
     
   \begin{flushleft}\raggedleft #1\end{flushleft}
}

%%%%%%%%%%%%%%%%%%%%%%%%%%%%%%%%%%%%%%%%%%%%%%%%%%%%%%%%%%%%%%%%%%%%%%%%%%%%%%%%

% TODO: remove when done
\definecolor{MyRed}{rgb}{0.8,0.05,0.05}
\definecolor{MyOrange}{rgb}{0.8,0.45,0}
\definecolor{MyGreen}{rgb}{0.13,0.65,0.13}
% 1) Red — there is no plan of content
\newcommand{\todored}[1]{\textcolor{MyRed}{#1}}
% 2) Orange — there is an outline/plan, needs to be written
\newcommand{\todoorange}[1]{\textcolor{MyOrange}{#1}}
% 3) Green — content is ready, needs review and some writing improvements
\newcommand{\todogreen}[1]{\textcolor{MyGreen}{#1}}

%%%%%%%%%%%%%%%%%%%%%%%%%%%%%%%%%%%%%%%%%%%%%%%%%%%%%%%%%%%%%%%%%%%%%%%%%%%%%%%%

\begin{document}

% Add title (yes, it' separate).
\includepdf[fitpaper=true,noautoscale=true]{title.pdf}

\tableofcontents
% \listoftables
\listoffigures

\newpage

\begin{abstract}

В настоящей работе описан, реализован и оценен алгоритм вывода типов в стиле Хиндли–Милнера, дополненный let-обобщением через уровни для обобщённого абстрактного синтаксиса со связанными переменными. Произведён анализ классической системы HM, механизмов уровневого обобщения и подхода Free Foil для безопасного управления областями видимости переменных. Практическая часть включает спецификацию грамматики целевого языка и предложенного алгоритма, генерацию обобщённого безопасного для областей видимости АСД, реализацию разделённых фаз сбора и решения ограничений, а также проверку корректности реализованного алгоритма через тестовый набор. Полученные результаты служат проверкой концепции, а в заключении работы предложены направления для дальнейшего развития, в частности, по обобщению алгоритма вывода типов и вынесению его в отдельную библиотеку.
\end{abstract}


\setcounter{page}{6}
\chapter{Introduction}
\label{chap:intro}

Type system is a crucial part of any formal system, such as a programming language or a proof assistant. In a world of functional programming, discovery of the Hindley-Milner (HM) type system played an important role of the field advancement. Although easy to implement, the classical algorithm $\mathcal{W}$~\cite{Milner1978_TypePolymorphism} is able to infer the principal type scheme~\cite{Hindley1969_PrincipalTypeScheme} of a term without any hints from a programmer. Moreover, both its soundness and completeness have been theoretically proved~\cite{Damas1984_TypeAssignment}. Ideas discovered by Hindley, Milner, and Damas are at the core of type systems of today's most powerful functional programming languages like ML, OCaml, and Haskell.

Definitely, numerous implementations of the Hindley-Milner type system can be found in both open-source projects and academic literature. Besides original algorithms $\mathcal{W}$~and~$\mathcal{J}$~\cite{Milner1978_TypePolymorphism} introduced in late 1978, other variations of HM-style inference with different goals have arised in recent decades, such as with deferred constraint solving (Heeren \textit{et al.}, 2002~\cite{Heeren2002_GeneralizingHM}), or with coercive subtyping (Traytel \textit{et al.}, 2011~\cite{Traytel2011_HMCoerciveSubtyping}).

Despite the abundance of existing variations and examples, building a correct, efficient, and maintainable implementation of the Hindley-Milner type system, especially with the prospect of future extension, remains a challenging task. For instance, let-generalization in $\mathcal{W}$ is done by traversing the whole type environment~\cite{DamasMilner1982_TypeSchemes}, which can be inefficient. An alternative rank-based (or level-based) generalization approach, introduced by R\'emy~\cite{Remy1992_SortedEqTheoryTypes}, has been employed in OCaml programming language~\cite{Kiselyov2022_OCamplTypeChecker} to achieve better efficiency, though it remains relatively obscure.

Scopes management and capture-avoiding substitution are another obstacles faced by type checker implementors. Efficient and correct implementation of abstract syntax with binders is painless, that is why Google research team developed "foil"~\cite{Foil} — a technique for achieving scope-safe and fast substitutions for scoped AST. Free Foil~\cite{FreeFoil} combines foil with data types a l\'a carte approach~\cite{Swierstra2008_a_la_carte}, to make foil's efficient and scope-safe representation of syntax also generic. Generic representation makes it easy to extend functionality with recursive operations such as traversal, or alpha equivalence check.

We believe that type checking can be a generic and modular component, enabling prototyping of new type systems easier, in a similar way how Free Foil simplifies adding support of binders to a language.

This work makes a small step towards the belief with the following contributions. First, we demonstrate an implementation of a Hindley-Milner-style type inference algorithm in Haskell, which is based on scope-safe abstract syntax generated with Free~Foil~\cite{FreeFoil}. Second, we adopt the idea of level-based type generalization to the inference. Third, we test the correctness of the algorithm on a test suite of more than 50 test cases with both well and ill typed sets of programs. Finally, we propose further steps for creating a generic version of the implemented algorithm, which could be used as a flexible type inference library once created.

Source code of this and related works of my colleagues is avilable at\newline \href{https://github.com/evermake/free-foil-typecheck}{github.com/evermake/free-foil-typecheck}.

\chapter{\todoorange{Literature Review}}
\label{chap:lr}

%%%%%%%%%%%%%%%%%%%%%%%%%%%%%%%%%%%%%%%%%%%%%%%%%%%%%%%%%%%%%%%%%%%%%%%%%%%%%%%%
% - Identify groups of related work (e.g. variations of an algorithm, implementations, theoretical results, etc.)
% - This section may include papers (published research, pre-prints), existing libraries, software, blog posts, etc.
% - You should not copy-paste abstract of a reviewed paper. Instead, you must read it carefully and identify the key points of this work that relate to your thesis.
% - To cover:
%   - системы типов и HM
%   - раздел про free-foil с описанием (главное: про что он и как им пользоваться)

%%%%%%%%%%%%%%%%%%%%%%%%%%%%%%%%%%%%%%%%%%%%%%%%%%%%%%%%%%%%%%%%%%%%%%%%%%%%%%%%

\section{\todoorange{Hindley-Milner Type System}}

In 1969, Hindley discovered an approach of computing a principal type scheme for a given object in combinatory logic \cite{Hindley1969_PrincipalTypeScheme}. Later, in 1978, Milner introduced two algorithms for inferring polymorphic type for a given program: algorithm $\mathcal{W}$ and $\mathcal{J}$, and proved soundness of $\mathcal{W}$ \cite{Milner1978_TypePolymorphism}. Finally, in 1984 \cite{Damas1984_TypeAssignment}, Damas proved completeness of $\mathcal{W}$ and that $\mathcal{W}$ indeed infers the principal type of a given term. Mentioned works formed the theoretical basis for the type system khown as Hindley-Milner (HM) type system in the literature. HM's ...

Hindley's "principal type scheme" and Milner's "polymorphic type" mean 

<... and known in literature as Hindley-Milner (HM) type system ...>

<explain constraints, unification and substitution>

<explain let-polymorphism>

<show typing rules and introduce the efficiency problem of type generalization process>

\begin{figure}[H]
  \begin{mathpar}
    \inferrule
      {\tau = \texttt{newvar} \quad
      \Gamma, x : \tau \vdash e : \tau'}
      {\Gamma \vdash \lambda x. e : \tau \to \tau'}
      \,[\texttt{Abs}]

    \inferrule
      {\Gamma \vdash e_0 : \tau_0 \quad
      \Gamma \vdash e_1 : \tau_1 \quad
      \tau' = \texttt{newvar} \quad
      \texttt{unify}(\tau_0, \tau_1 \to \tau')}
      {\Gamma \vdash e_0\,e_1 : \tau'}
      \,[\texttt{App}]

    \inferrule
      {\Gamma \vdash e_0 : \tau \quad
      \Gamma, x : \overline{\Gamma}(\tau) \vdash e_1 : \tau'}
      {\Gamma \vdash \texttt{let } x = e_0 \texttt{ in } e_1 : \tau'}
      \,[\texttt{Let}]
    
    \inferrule
      {x : \sigma \in \Gamma \quad
      \tau = \texttt{inst}(\sigma)}
      {\Gamma \vdash x : \tau}
      \,[\texttt{Var}]
  \end{mathpar}
  \caption{Algorithm $\mathcal{J}$ typing rules.}
\end{figure}

\begin{figure}[H]
  \begin{mathpar}
    \overline{\Gamma}(\tau) = \forall \hat{\alpha}.\tau

    \hat{\alpha} = \texttt{free}(\tau) \setminus \texttt{free}(\Gamma)  
  \end{mathpar}
  \caption{Generalization of a type $\tau$ under context $\Gamma$.}
\end{figure}

%%%%%%%%%%%%%%%%%%%%%%%%%%%%%%%%%%%%%%%%%%%%%%%%%%%%%%%%%%%%%%%%%%%%%%%%%%%%%%%%
%%%%%%%%%%%%%%%%%%%%%%%%%%%%%%%%%%%%%%%%%%%%%%%%%%%%%%%%%%%%%%%%%%%%%%%%%%%%%%%%
%%%%%%%%%%%%%%%%%%%%%%%%%%%%%%%%%%%%%%%%%%%%%%%%%%%%%%%%%%%%%%%%%%%%%%%%%%%%%%%%

\section{\todoorange{Levels}}

In HM type system, let-polymorphism enables the definition of polymorphic functions through let-expressions. The process involves two primary steps: generalization and specialization (or instantiation). Generalization quantifies free type variables in the monotype of a let-bound expression to form a polytype, while specialization instantiates these quantified variables with fresh ones at each use of the bound variable.

However, generalizing type variables introduced outside the let binding can lead to unsound type checking. Consider the expression:

$$
\lambda x.\ \texttt{let}\ y = (\lambda z.\ x)\ \texttt{in}\ y
$$

Here, the type of expression bound by $y$ is inferred as $A \to B$, where $B$ and $A$ are the types of $x$ and $z$ accordingly. Generalizing both $A$ and $B$ to form $\forall \alpha. \forall \beta.\ \alpha \to \beta$ would be incorrect, as $\beta$ is bound in the outer context. The correct type for $y$ is $\forall \alpha.\ \alpha \to B$, ensuring that only $\alpha$, introduced within the expression bound by let, is generalized.

Traditional type inference algorithms, such as $\mathcal{W}$~and~$\mathcal{J}$~\cite{Milner1978_TypePolymorphism}, perform generalization by computing the free type variables in the inferred type and subtracting those free in the type environment. This process requires scanning both the type and the entire environment. To address this inefficiency, Didier R\'emy introduced the concept of ranks (or levels) in his work~\cite{Remy1992_SortedEqTheoryTypes}. As pointed out by R\'emy~\cite{Remy1992_SortedEqTheoryTypes}, this inefficiency is not particularly critical in the standard HM, but becomes important when new extensions, such as subtyping, are introduced.

Levels are integers representing the nesting depth of let-expressions. Each type variable is associated with a level corresponding to the let-expression in which it was introduced, with level 1 assigned to the implicit top-level let of a program. Level assigned to a unification variable may change after it was introduced. During unification, if a type variable is unified with a type at a lower level, its level is updated to the minimum of the two, ensuring it reflects the outermost scope in which it is used. Finally, when generalizing a let-bound expression at level $n$, only free variables of the type with levels greater than $n$ are quantified. As a result, sound generalization is preserved without "touching" a typing environment at all.

<TODO: demonstrate on example above...>

%%%%%%%%%%%%%%%%%%%%%%%%%%%%%%%%%%%%%%%%%%%%%%%%%%%%%%%%%%%%%%%%%%%%%%%%%%%%%%%%
%%%%%%%%%%%%%%%%%%%%%%%%%%%%%%%%%%%%%%%%%%%%%%%%%%%%%%%%%%%%%%%%%%%%%%%%%%%%%%%%
%%%%%%%%%%%%%%%%%%%%%%%%%%%%%%%%%%%%%%%%%%%%%%%%%%%%%%%%%%%%%%%%%%%%%%%%%%%%%%%%

\section{\todogreen{Name Management}}

An integral feature of any programming language or formal system is an ability to work with bound names (or local variables), e.g. a parameter of a $\lambda$-abstraction. In the example below, $x$ is said to be \textit{bound} by a \textit{binder} $\lambda x.$, and $y$ is said to be \textit{free} as it does not have a binder:

$$
\lambda x." "x" "y
$$

The main pain point of name management is scoping, that is keeping track of which names are bound by which binders in order to resolve names correctly avoiding collisions. The common operation of name management is substitution, that is, replacing the occurances of a variable with another expression. We may want to replace the occurances of the variable bound by $\lambda$-abstraction, in order to reduce the expression. Or, during type inference, when specializing a polymorphic type in let expression, it is necessary to substitute quantified type variables with the fresh ones. Both these tasks, as well as many others, require a substitution. However, implementing a substitution is not trivial at all.

To illustrate why substitution is not an easy task, consider the process of reduction of this $\lambda$-expression:

\begin{align*}
  & (\lambda x. \lambda y. x) y \\
  & \rightarrow \text{($\beta$-reduction)} \\
  & [x \mapsto y](\lambda y. x) \\
  & \rightarrow \text{(substitution)} \\
  & \textcolor{red}{\lambda y. y}
\end{align*}

We got an identity function, which is wrong, since $y$ in the body of $\lambda$-abstraction is not the same $y$ as before, it has been \textit{captured}. Naïve substitution does not work in this case. Instead, we need a \textit{capture-avoiding substitution}.

Capture-avoiding substitution is an old and well-studied problem in computer science, with extensive research done as well as numerous practical implementations applied in real-world projects.

De~Bruijn indices~\cite{deBruijn1972} is one of the earliest and simplest approaches to capture-avoiding substitution, which suggests to replace bound names with natural numbers representing the position relative to their binder. However, such nameless representation is not efficient in practice due to often shifts and expressions traversal. In recent decades, more efficient approaches to capture-avoiding substitution have emerged. One notable example is the stateless "rapier" technique~\cite{Simon2002_SecretsGHC}, which is used by the Glasgow Haskell Compiler and has proven to be highly performant. <TODO: Maybe reference to some benchmark?>

However, besides efficency, it is worth considering another two important aspects when choosing between different name management techniques, which are intrinsic scoping and generality.

\section{\todoorange{Intrinsic Scoping}}

Intrinsically scoped, or scope-safe, abstract syntax representation of the language terms helps to minimize a chance of introducing bugs, by leveraging the power of the host language type system. For example, in 1999 Bird and Paterson \cite{BirdPaterson1999_BruijnNested} showed how invariants of the de Bruijn notation can be described on the type level by using nested datatypes \cite{Bird1998_NestedDatatypes}. Dex programming language~\cite{PaszkeDex_2021} is another illustrative example why using scope-safe abstract syntax is beneficial. When adopting the mentioned "rapier" technique in Dex, Google research team faced many difficulties, which led them to develop the foil~\cite{Foil} — an enhancement that leverages the Haskell's type system to maintain rapier's invariants.

<TODO: explain how scope-safety is achieved in foil>

\section{\todoorange{Generic Abstract Syntax}}

Generality is another possible characteristic of the abstract syntax, which helps to avoid repeatition and code duplication. With generic abstract syntax it becomes possible to reuse already implemented and tested algorithms, as well as to easily extend existing code. 

%%%%%%%%%%%%%%%%%%%%%%%%%%%%%%%%%%%%%%%%%%%%%%%%%%%%%%%%%%%%%%%%%%%%%%%%%%%%%%%%
%%%%%%%%%%%%%%%%%%%%%%%%%%%%%%%%%%%%%%%%%%%%%%%%%%%%%%%%%%%%%%%%%%%%%%%%%%%%%%%%
%%%%%%%%%%%%%%%%%%%%%%%%%%%%%%%%%%%%%%%%%%%%%%%%%%%%%%%%%%%%%%%%%%%%%%%%%%%%%%%%

\section{\todoorange{Free Foil}}

Free Foil introduced by Kudasov \textit{et al.} \cite{FreeFoil}, is a framework that generates efficient, scope-safe abstract syntax, enabling users to effortlessly add support of name management into an object language.

...

<TODO: demonstrate main data types and ideas of free foil>

...

Scope-safe generic abstract syntax can be generated with Free Foil in two ways: by using free scoped monads or via Template Haskell \cite{SheardPeytonJones2002_TH} GHC extension, i.e. metaprogramming. In this work we have chosen the second approach, as it requires writing less code and, more importantly, integrates nicely with BNFC.

%%%%%%%%%%%%%%%%%%%%%%%%%%%%%%%%%%%%%%%%%%%%%%%%%%%%%%%%%%%%%%%%%%%%%%%%%%%%%%%%
%%%%%%%%%%%%%%%%%%%%%%%%%%%%%%%%%%%%%%%%%%%%%%%%%%%%%%%%%%%%%%%%%%%%%%%%%%%%%%%%
%%%%%%%%%%%%%%%%%%%%%%%%%%%%%%%%%%%%%%%%%%%%%%%%%%%%%%%%%%%%%%%%%%%%%%%%%%%%%%%%

\section{\todogreen{Code Generation Tools}}

Backus–Naur Form Converter (BNFC) \cite{BNFC} is a tool for generating compiler front-end for an object language, given its' grammar in Labelled BNF \cite{BackusNaurForm2003}. By assigning a label to each rule in the grammar, we tell the BNFC what constructors to use when generating the syntax tree in the host language.

Figure~\ref{fig:lbnf-bnfc-example} below illustrates how BNF rules and their labels defined in the grammar file correspond to the Haskell code generated by BNFC. Different rules with the same label correspond to the different constructors of the same type. In the generated code, terminals from the grammar (such as \texttt{"true"} in the \texttt{ETrue} rule) are omitted, while non-terminals (such as \texttt{Exp1} and \texttt{Exp2} in the \texttt{EAdd} rule) become parameters of the corresponding type constructor. Lastly, integer postfix in the name of each rule only serves as the precedence level of this rule, and is ommitted in the resulting type name. Thus, both \texttt{Exp1} and \texttt{Exp2} are merged to \texttt{Exp} in code.

\begin{figure}[H]
  \centering
  \begin{minipage}{0.49\textwidth}
    \begin{minted}[frame=single,fontsize=\small]{text}
EVar.    Exp2 ::= Ident ;
ETrue.   Exp2 ::= "true" ;
EFalse.  Exp2 ::= "false" ;
ENat.    Exp2 ::= Integer ;
EAdd.    Exp1 ::= Exp1 "+" Exp2 ;
    \end{minted}
    \hfill
  \end{minipage}
  \begin{minipage}{0.49\textwidth}
    \begin{minted}[frame=single,fontsize=\small]{haskell}
data Exp
  = EVar Ident
  | ETrue 
  | EFalse
  | ENat Integer
  | EAdd Exp Exp
    \end{minted}
  \end{minipage}
  \caption[Input grammar and code produced by BNFC]{Example of the LBNF grammar and corresponding data type in Haskell, generated by BNFC.}
  \label{fig:lbnf-bnfc-example}
\end{figure}

BNFC significantly simplifies the process of prototyping a compiler, or a type checker for an object language. By supplying only the language grammar to BNFC, user gets a handful of functionality generated automatically, such as abstract syntax with pretty-printing, and, more importantly, specifications for lexer and parser generators. In fact, to further generate code of the lexer and parser, additional language-specific tools are required. And since the focus of this work is to implement a type checker in \textit{Haskell}, we use BNFC together with Alex \cite{haskell_alex}, a lexer generator, and Happy \cite{haskell_happy}, a parser generator.

\chapter{Методология}
\label{chap:met}

Методология работы включает формализацию языка, определение алгоритма $\mathcal{L}$ и описание вспомогательных функций.

Целевой язык — лямбда-исчисление с натуральными и булевыми литералами и let-выражениями. Грамматика выражений и типов представлена на рисунке~\ref{fig:object-language-grammar}.

\begin{figure}[H]
  \fbox{\begin{minipage}{\dimexpr\textwidth-2\fboxsep-2\fboxrule}%
    \begin{center}
      \begin{grammar}
        \firstcase{\text{$e$}}{0 | 1 | \ldots}{constant natural number}
        \otherform{\texttt{true}}{constant true}
        \otherform{\texttt{false}}{constant false}
        \otherform{x}{variable}
        \otherform{\lambda x . e}{abstraction}
        \otherform{e\;e}{application}
        \otherform{\texttt{let } x \texttt{ = } e \texttt{ in } e} {let-binding}

        \firstcase{\text{$\tau$}}{\texttt{Nat}}{natural number type}
        \otherform{\texttt{Bool}}{boolean type}
        \otherform{\alpha}{type variable}
        \otherform{\tau \to \tau}{function (arrow)}
        \otherform{\forall \alpha. \tau}{forall}
      \end{grammar}
    \end{center}
  \end{minipage}}
  \caption{Грамматица целевого языка}
  \label{fig:object-language-grammar}
\end{figure}

Алгоритм $\mathcal{L}$ основан на манипуляции контекста $\mathcal{T} = (C,\mathcal{S},\Gamma,M,\ell)$, где:
\begin{itemize}
  \item $C$ — набор ограничений;
  \item $\mathcal{S}$ — текущая подстановка;
  \item $\Gamma$ — среда типизации;
  \item $M$ — отображение уровней переменных;
  \item $\ell$ — текущий уровень вложенности.
\end{itemize}

Математическое описание предложенного алгоритма $\mathcal{L}$ представлено на рисунке~\ref{fig:algorithm-L}.


\begin{figure}[H]
  \fbox{\begin{minipage}[c][\textheight-2\fboxsep-2\fboxrule-1cm][c]{\dimexpr\textwidth-2\fboxsep-2\fboxrule}%
    \begin{alignat*}{4}
      % Literals
      & \mathcal{L} (\mathcal{T}_1, n) &&= & &(\mathcal{T}_1, \texttt{Nat}), \text{where } n \in \mathbb{N}\\
      & \mathcal{L} (\mathcal{T}_1, \texttt{true}) &&= & &(\mathcal{T}_1, \texttt{Bool}) \\
      & \mathcal{L} (\mathcal{T}_1, \texttt{false}) &&= & &(\mathcal{T}_1, \texttt{Bool}) \\
      % Variable
      & \mathcal{L} ((C_1, \mathcal{S}_1, \Gamma_1, M_1, \ell_1), x) &&= &\textbf{ let } &(\tau_1, \vec{\alpha}) = \mathrm{spec}(\tau_0), \text{where } (x : \tau_0) \in \Gamma_1 \\
      & && & &M_2 = M_1 \union \{ \alpha \mapsto \ell_1 | \alpha \in \vec{\alpha} \}\\
      & && &\textbf{in } &((C_1, \mathcal{S}_1, \Gamma_1, M_2, \ell_1), \tau_1)\\
      % Abstraction
      & \mathcal{L} ((C_1, \mathcal{S}_1, \Gamma_1, M_1, \ell_1), \lambda x. e) &&= &\textbf{ let } &\Gamma_2 = \Gamma_1 \union \{(x : \beta)\}, \beta \text{ is fresh} \\
      & && & & M_2 = M_1 \union \{\beta \mapsto \ell_1\} \\
      & && & & (\mathcal{T}_3, \tau_0) = \mathcal{L}((C_1, \mathcal{S}_1, \Gamma_2, M_2, \ell_1), e) \\
      & && &\textbf{in } &(\mathcal{T}_3, \beta \to \tau_0) \\
      % Application
      & \mathcal{L} (\mathcal{T}_1, e_1\;e_2) &&= &\textbf{ let } &(\mathcal{T}_2, \tau_1) = \mathcal{L}(\mathcal{T}_1, e_1) \\
      & && & &((C_3, \mathcal{S}_3, \Gamma_3, M_3, \ell_3), \tau_2) = \mathcal{L}(\mathcal{T}_2, e_2)\\
      & && & & C_4 = C_3 \union \{\tau_1 \equiv \tau_2 \to \beta\}, \beta \text{ is fresh} \\
      & && & & M_4 = M_3 \union \{\beta \mapsto \ell_3\} \\
      & && &\textbf{in } &((C_4, \mathcal{S}_3, \Gamma_3, M_4, \ell_3), \beta) \\
      % Let
      & \mathcal{L} (\mathcal{T}_1, \texttt{let } x \texttt{ = } e_1 \texttt{ in } e_2) &&= &\textbf{ let } & (C_1, \mathcal{S}_1, \Gamma_1, M_1, \ell_1) = \mathcal{T}_1 \\
      & && & &(\mathcal{T}_2, \tau_1) = \mathcal{L}((C_1, \mathcal{S}_1, \Gamma_1, M_1, \ell_1 + 1), e_1)\\
      & && & &(C_2, \mathcal{S}_2, \Gamma_2, M_2, \ell_2) = \mathcal{T}_2\\
      & && & &(\mathcal{S}_3, M_3) = \mathrm{unify}(C_2, \mathcal{S}_2, M_2)\\
      & && & &\Gamma_3 = \mathcal{S}_3\Gamma_2 \union \{(x:\mathrm{gen}(\mathcal{S}_3 \tau_1, M_3, \ell_1))\}\\
      & && &\textbf{in } &\mathcal{L}((\varnothing, \mathcal{S}_3, \Gamma_3, M_3, \ell_1), e_2) \\
    \end{alignat*}
  \end{minipage}}
  \caption{Алгоритм $\mathcal{L}$}\label{fig:algorithm-L}
\end{figure}

\chapter{\todoorange{Implementation}}
\label{chap:implementation}

In this chapter, we present an implementation of the mentioned type inference algorithm in Haskell. First, we define object language grammar and generate front-end of the type checker. Second, we generate generic abstract syntax with Free Foil. Then, we introduce necessary data types and explain key steps in the algorithm. Finally, we talk about testing and related difficulties.

\section{\todogreen{Grammar and Front-end}}

The object language, for which the type checking is to be implemented is similar to the simply typed lambda-calculus but extended with let polymorphism, boolean and natural number literals. The grammar is defined in Labelled BNF (LBNF), and is divided into two parts: terms (or expressions) and types.

%%%%%%%%%%%%%%%%%%%%%%%%%%%%%%%%%%%%%%%%%%%%%%%%%%%%%%%%%%%%%%%%%%%%%%%%%%%%%%%%

\subsection{\todogreen{Expressions Grammar}}

The LBNF grammar for expressions is shown in Figure~\ref{fig:lbnf-terms}. We define 7 grammar rules for expressions:

\begin{itemize}
  \item \texttt{EVar} — variable identifier.
  \item \texttt{ETrue} and \texttt{EFalse} — boolean literals.
  \item \texttt{ENat} — natural number literal.
  \item \texttt{EApp} — function application. Note that subexpression on the right has a higher precedence level, since function application is left associative.
  \item \texttt{EAbs} — lambda abstraction, which binds a variable to be used in the nested expression.
  \item \texttt{ELet} — let-binding, which is similar to \texttt{EAbs} but has one more (not scoped) subexpression.
\end{itemize}

\texttt{Pattern} and \texttt{ScopedExp} rules are essentially wrappers around \texttt{Ident} and \texttt{Exp1} respectively, which are introduced to clearly identify a binder and corresponding scoped expressions where the bound pattern can occur. These rules are crucial for the scope-safe abstract syntax generation, which is described in the next section.

\begin{figure}[H]
  \begin{minted}[frame=single,fontsize=\small,escapeinside=@@]{text}
EVar.    Exp3 ::= Ident ;
ETrue.   Exp3 ::= "true" ;
EFalse.  Exp3 ::= "false" ;
ENat.    Exp3 ::= Integer ;
EApp.    Exp2 ::= Exp2 Exp3 ;
EAbs.    Exp1 ::= "@$\lambda$@" Pattern "." ScopedExp ;
ELet.    Exp1 ::= "let" Pattern "=" Exp1 "in" ScopedExp ;
coercions Exp 3 ;

PatternVar. Pattern ::= Ident ;
ScopedExp. ScopedExp ::= Exp1 ;
  \end{minted}
  \caption[LBNF grammar of the object language expressions]{Labelled BNF grammar of the object language expressions. \texttt{Ident} and \texttt{Integer} are predifined basic types in BNFC. The \texttt{coercions} is a macro that automatically generates semantic "no-op" coercion rules that define the precedence of the operators.}
  \label{fig:lbnf-terms}
\end{figure}

%%%%%%%%%%%%%%%%%%%%%%%%%%%%%%%%%%%%%%%%%%%%%%%%%%%%%%%%%%%%%%%%%%%%%%%%%%%%%%%%

\subsection{\todogreen{Types Grammar}}

Figure~\ref{fig:lbnf-types} shows the LBNF grammar for types, which defines 6 grammar rules:

\begin{itemize}
  \item \texttt{TUVar} — free, or unification variable.
  \item \texttt{TBool} — boolean type.
  \item \texttt{TNat} — natural number type.
  \item \texttt{TVar} — type variable.
  \item \texttt{TArrow} — function type. Unlike function expressions, function types are right associative, therefore, rule to the left of the arrow has higher precedence.
  \item \texttt{TForAll} — "forall"-quantified type that binds a type variable in the inner type expression.
\end{itemize}

To distinguish between free and bound variables we prefix unification variable identifiers with "?" by introducing a new \texttt{UVarIdent} lexical type. And, similarly to term expressions, we also define separate \texttt{TypePattern} and \texttt{ScopedType} rules to further generate the scope-safe abstract syntax for types. In case of types, the only rule that binds a (type) variable and has a scoped subexpression is \texttt{TForAll}.

\begin{figure}[H]
  \begin{minted}[frame=single,fontsize=\small,escapeinside=@@]{text}
token UVarIdent ('?' letter (letter | digit | '_')*) ;

TUVar.   Type2 ::= UVarIdent ;
TNat.    Type2 ::= "Nat" ;
TBool.   Type2 ::= "Bool" ;
TVar.    Type2 ::= Ident ;
TArrow.  Type1 ::= Type2 "->" Type1 ;
TForAll. Type  ::= "forall" TypePattern "." ScopedType ;
coercions Type 2 ;

TPatternVar. TypePattern ::= Ident ;
ScopedType. ScopedType ::= Type1 ;
  \end{minted}
  \caption[LBNF grammar of the object language types]{Labelled BNF grammar of the object language types. \texttt{UVarIdent} is a custom lexical type introduced to distinguish unification (free) variables from bound variables, which use the predifined basic type \texttt{Ident}. The \texttt{coercions} is a macro that automatically generates semantic "no-op" coercion rules that define the precedence of the operators.}
  \label{fig:lbnf-types}
\end{figure}

%%%%%%%%%%%%%%%%%%%%%%%%%%%%%%%%%%%%%%%%%%%%%%%%%%%%%%%%%%%%%%%%%%%%%%%%%%%%%%%%

\subsection{\todogreen{Code Generation}}

Having defined the language grammar, we are ready to generate the necessary Haskell code using code generation tools mentioned in the Chapter~\ref{chap:lr}. After installing necessary tools, and running shell commands from Figure~\ref{fig:code-gen-cli}, we get:

\begin{enumerate}
  \item Abstract syntax data types (\texttt{Abs.hs}), pretty-printing (\texttt{Print.hs}), lexer specification (\texttt{Lex.x}), and parser specification (\texttt{Par.y}) generated by BNFC~\cite{BNFC} from the grammar.
  \item Lexical analyzer (\texttt{Lex.hs}) generated by Alex~\cite{haskell_alex} from the lexer specification.
  \item Parser (\texttt{Par.hs}) generated by Happy~\cite{haskell_happy} from the parser specification.
\end{enumerate}

\begin{figure}[H]
  \begin{minted}[frame=single]{bash}
bnfc --haskell -d \
  -p FreeFoilTypecheck.HindleyMilner \
  --generic \
  -o src \
  grammar/Parser.cf
alex src/FreeFoilTypecheck/HindleyMilner/Parser/Lex.x
happy src/FreeFoilTypecheck/HindleyMilner/Parser/Par.y
  \end{minted}
  \caption{Shell commands for generating type checker front-end}
  \label{fig:code-gen-cli}
\end{figure}

\begin{figure}[H]
\begin{minted}[frame=single,fontsize=\small]{haskell}
newtype Ident = Ident String

data Pattern = PatternVar Ident
data Exp
  = EVar Ident
  | ETrue
  | EFalse
  | ENat Integer
  | EApp Exp Exp
  | ELet Pattern Exp ScopedExp
  | EAbs Pattern ScopedExp
data ScopedExp = ScopedExp Exp

newtype UVarIdent = UVarIdent String
data TypePattern = TPatternVar Ident
data Type
  = TUVar UVarIdent
  | TNat
  | TBool
  | TVar Ident
  | TArrow Type Type
  | TForAll TypePattern ScopedType
data ScopedType = ScopedType Type
\end{minted}
  \caption{Abstract syntax data types generated by BNFC}
  \label{fig:ast-types-bnfs}
\end{figure}

From the abstract syntax generated by BNFC showon on Figure~\ref{fig:ast-types-bnfs}, we now are able to generate the generic and scope-safe version of it via Free Foil.

\section{\todogreen{Generating Scope-Safe SOAS}}

The Free Foil framework is accessible in Haskell via the free-foil package~\cite{free-foil}. To generate scope-safe second-order representation of the syntactical constructs of our language, Free Foil requires the first-order syntax to be organized in four types~\cite{FreeFoil}: \textit{variable identifiers}, \textit{patterns} (binders), \textit{scoped terms}, and (unscoped) \textit{terms}. This is exactly what we have generated with BNFC (Figure~\ref{fig:ast-types-bnfs}). In total, we supply free-foil with 7 types to generate SOAS, since we use expressions and types separately:

\begin{enumerate}
  \item \texttt{Ident} — type of \textit{variable identifier} in both expressions and types.
  \item \texttt{Pattern} — type of \textit{pattern} in expressions.
  \item \texttt{ScopedExp} — type of \textit{scoped terms} in expressions.
  \item \texttt{Exp} — type of \textit{terms} in expressions.
  \item \texttt{TypePattern} — type of \textit{patterns} in types.
  \item \texttt{ScopedType} — type of \textit{scoped terms} in types.
  \item \texttt{Type} — type of \textit{terms} in types.
\end{enumerate}

Code on Figure~\ref{fig:soas-gen} generates signature for type expressions, necessary class instances, and other helpers to be used later. \texttt{Type n} is a type synonym with phantom parameter \texttt{n}, which represents a single AST node in scope \texttt{n}, and is used throughout the implementation. \texttt{Type'} is a synonym for a type expression in empty scope, i.e. in the scope where all variables (if any) are bound.

\begin{figure}[H]
\begin{minted}[frame=single,fontsize=\small]{haskell}
-- Signature
mkSignature ''Raw.Type ''Raw.Ident ''Raw.ScopedType ''Raw.TypePattern
deriveZipMatch ''TypeSig
deriveBifunctor ''TypeSig
deriveBifoldable ''TypeSig
deriveBitraversable ''TypeSig

-- Pattern synonyms
mkPatternSynonyms ''TypeSig

--Conversion helpers
mkConvertToFreeFoil ''Raw.Type ''Raw.Ident ''Raw.ScopedType ''Raw.TypePattern
mkConvertFromFreeFoil ''Raw.Type ''Raw.Ident ''Raw.ScopedType ''Raw.TypePattern

-- Scope-safe type patterns
mkFoilPattern ''Raw.Ident ''Raw.TypePattern
deriveCoSinkable ''Raw.Ident ''Raw.TypePattern
mkToFoilPattern ''Raw.Ident ''Raw.TypePattern
mkFromFoilPattern ''Raw.Ident ''Raw.TypePattern

type Type n = AST FoilTypePattern TypeSig n
type Type' = Type Foil.VoidS
\end{minted}
  \caption{SOAS generation with free-foil's Template Haskell functions}
  \label{fig:soas-gen}
\end{figure}

\section{\todoorange{Inference Algorithm}}

<TODO: описываю детали реализации ”для реализации мы выбрали простые типы данных (строки, map) и не используем эффективные библиотеки — можно аргументировать: потому что реализация “self-contained“, т.е. она простая, но зато её легко понять”>

\begin{figure}[H]
  \begin{minted}[frame=single,fontsize=\small]{haskell}
type IdentLevelMap = HashMap.HashMap Raw.UVarIdent Int

instance Data.Hashable.Hashable Raw.UVarIdent where
  hashWithSalt salt (Raw.UVarIdent s) = Data.Hashable.hashWithSalt salt s

newtype Constraint n = Constraint (Type n, Type n)
type Constraint' = Constraint Foil.VoidS

newtype Subst n = Subst (Map.Map Raw.UVarIdent (Type n))
type Subst' = Subst Foil.VoidS

newtype TypingEnv n tn = TypingEnv (Foil.NameMap n (Type tn))
type TypingEnv' n = TypingEnv n Foil.VoidS

class Typed a where
  applySubst :: (Foil.Distinct n) => Subst n -> a n -> a n

  freeVars :: a n -> Set.Set Raw.UVarIdent

  hasFreeVar :: Raw.UVarIdent -> a n -> Bool
  hasFreeVar i t = i `Set.member` freeVars t

instance Typed (FreeFoil.AST FoilTypePattern TypeSig) where
  applySubst (Subst s) t = foldr applySubstToType t (Map.toList s)

  freeVars (TUVar ident) = Set.singleton ident
  freeVars (FreeFoil.Var _) = Set.empty
  freeVars (FreeFoil.Node node) = Set.unions $ freeVars <$> node

instance Typed Constraint where
  applySubst s (Constraint (t1, t2)) = Constraint (applySubst s t1, applySubst s t2)

  freeVars (Constraint (t1, t2)) = Set.union (freeVars t1) (freeVars t2)

instance Typed (TypingEnv n) where
  applySubst s (TypingEnv env) = TypingEnv (fmap (applySubst s) env)
  freeVars (TypingEnv env) = Set.unions $ freeVars <$> env

idSubst :: Subst Foil.VoidS
idSubst = Subst Map.empty

singleSubst :: (Foil.Distinct n) => Raw.UVarIdent -> Type n -> Subst n
singleSubst ident type_ = Subst (Map.singleton ident type_)

composeSubst :: (Foil.Distinct n) => Subst n -> Subst n -> Subst n
composeSubst (Subst s1) (Subst s2) = Subst (Map.map (applySubst (Subst s2)) s1 `Map.union` s2)
  \end{minted}
  \caption{todo}
  % \label{}
\end{figure}

...

\begin{figure}[H]
  \begin{minted}[frame=single,fontsize=\small]{haskell}
data TypingContext n = TypingContext
  { tcConstraints :: [Constraint'],
    tcSubst :: Subst',
    tcEnv :: TypingEnv' n,
    tcFreshId :: Int,
    tcLevelMap :: IdentLevelMap,
    tcLevel :: Int
  }

initialTypingContext :: TypingContext Foil.VoidS
initialTypingContext =
  TypingContext
    { tcConstraints = [],
      tcSubst = Subst Map.empty,
      tcEnv = TypingEnv Foil.emptyNameMap,
      tcFreshId = 1,
      tcLevelMap = HashMap.empty,
      tcLevel = 1
    }
  \end{minted}
  \caption{todo}
  % \label{}
\end{figure}

...

\texttt{inferType} takes an expression and returns monadic operation that performs type inference. More precisely, it takes a value of type \texttt{Exp n} and returns \texttt{TypeInferencer n Type'}, where:

\begin{itemize}
  \item \texttt{Exp n} is a synonym for \texttt{AST FoilPattern ExpSig n}, where
  \begin{itemize}
    \item \texttt{AST} — ???
    \item \texttt{FoilPattern} — ???
    \item \texttt{ExpSig} — ???
    \item \texttt{n} — ``phantom'' type parameter used required to avoid mixing different scopes on the type level.
  \end{itemize}
  \item \texttt{TypeInferencer n Type'} has two type parameters:
  \begin{itemize}
    \item \texttt{n} — S kind
    \item \texttt{Type'} — type of a payload that returned when type inferencer is computed, synonym for \texttt{AST FoilTypePattern TypeSig Foil.VoidS}
  \end{itemize}
\end{itemize}

How \texttt{inferType} works? It recursively infers a type of a given AST node (type \texttt{Exp n}) by pattern matching constructor of a node. For each node constructor, we define how type inference works for this specific node type. Let us consider the process for each type in detail:

\begin{itemize}
  \item \texttt{ETrue} and \texttt{EFalse} constructors do not have any parameters as they simply represent \texttt{true} and \texttt{false} boolean literals in the target language accordingly. \textless just \texttt{TBool}\textgreater
  \item \texttt{ENat} has one parameter of type \texttt{Integer} — value of the number literal, which does not affect the type inference and hence is ignored. \textless just \texttt{TNat}\textgreater
  \item \texttt{FreeFoil.Var} is special as
\end{itemize}

\section{\todored{Unification}}

We implement a standard unification algorithm according to Robinson~\cite{Robinson1965} in the \texttt{unifyConstraint} (Figure~\ref{fig:unifyConstraint}) function with two peculiarities.

First, we rely on free-foil's \texttt{zipMatch} function that unifies two arbitary AST nodes, which are then converted to constraints to further unify recursively with \texttt{unifyConstraints}. Since \texttt{zipMatch} is generic, unification works universally for all node types in AST.

Second, as our implementation of generalization relies on levels, level assigned to a type variable is updated during unification~\cite{Kiselyov2022_OCamplTypeChecker}, and since in our implementation a mapping is defined by \texttt{IdentLevelMap}, we need to update it.

\texttt{unifyConstraint} is splitted into two logical parts:

...

\begin{figure}[H]
  \begin{minted}[frame=single,fontsize=\small]{haskell}
unifyConstraint
  :: IdentLevelMap
  -> Constraint'
  -> Either String (Subst', IdentLevelMap)
unifyConstraint levelsMap (Constraint constr) =
  case constr of
    (TUVar x, r) -> case r of
      TUVar y
        | x == y -> Right (idSubst, levelsMap)
        | otherwise -> unifyWithUVar levelsMap x r
      _ -> unifyWithUVar levelsMap x r
    (l, TUVar x) -> unifyWithUVar levelsMap x l
    (FreeFoil.Node l, FreeFoil.Node r) ->
      case FreeFoil.zipMatch l r of
        Nothing -> Left ("cannot unify " ++ show constr)
        Just lr -> unifyConstraints levelsMap (Constraint <$> F.toList lr)
    (lhs, rhs) -> Left ("cannot unify " ++ show lhs ++ show rhs)
  \end{minted}
  \caption{...}
  \label{fig:unifyConstraint}
\end{figure}

...

\begin{figure}[H]
  \begin{minted}[frame=single,fontsize=\small]{haskell}
unifyWithUVar
  :: IdentLevelMap
  -> Raw.UVarIdent
  -> Type'
  -> Either String (Subst', IdentLevelMap)
unifyWithUVar levelsMap x type_
  | hasFreeVar x type_ = Left "occurs check failed"
  | otherwise = case HashMap.lookup x levelsMap of
      Nothing -> Left "type variable w/o level"
      Just level ->
        let vars = Set.toList (freeVars type_)
            withXLevels = HashMap.fromList [(var, level) | var <- vars]
            updatedLevels = HashMap.unionWith min levelsMap withXLevels
          in Right (singleSubst x type_, updatedLevels)
  \end{minted}
  \caption{...}
  % \label{}
\end{figure}

\section{\todogreen{Testing}}

A test suite consisting of 19 ill typed and 33 well typed programs has been developed when implementing the type inference. This helped to find bugs during development and, to some extent, verify the correctness of the implemented algorithm.

Test cases with ill typed programs check that inference algorithm indeed fails to infer type of such programs. For an ill typed program, test case passes only if the \texttt{inferTypeClosed} returns an error, for example, when unification of \texttt{TBool} with \texttt{TNat} is attempted, or when "occurs check" fails. We, however, do not check that the exact reason of failure is as expected for ill typed programs, which is left for the future work.

Test cases with well typed programs check that the inference algorithm successfully infers the type and it is correct. In order to check that the inferred type is correct, each well typed program is accompanied with additional file that contains the expected type of a program. For a well type program, test case passes only if the \texttt{inferTypeClosed} successfully returns a type that matches the expected type. By "matches" we mean that two polymorphic types are alpha-equivalent, except that the order of \texttt{forall} binders does not matter, in other words, we assume that types $\forall \alpha. \forall \beta. \alpha \to \beta$ and $\forall \beta. \forall \alpha. \alpha \to \beta$ are equal (which R\'emy called $\forall$-equality~\cite{Remy1992_SortedEqTheoryTypes}). Such assumption made it easier to add new test cases, as we do not need to think about implementation details when specifying expected types for the well typed programs.

It is worth mentioning how the comparison of an actual and expected types is implemented for a well typed test case, which is not straightforward. We could use the \texttt{alphaEquiv} function provided by the free-foil package for this purpose. However, it does not distinguish values of the literals (as illustrated in Figure~\ref{fig:freefoil-alphaequiv-literals}), and, therefore, is not suitable for comparing types containing free variables. A possible workaround would be to generalize both types before performing a comparison, as this would replace all free type variables with the variables bound by "forall"s. But still, this will not work in our case (see Figure~\ref{fig:freefoil-alphaequiv-foralls}), since we do not want the order of "forall"s in the types to influence the comparison. Therefore, free-foil's \texttt{alphaEquiv} does not fit our needs and another approach is necessary.

\begin{figure}[H]
  \begin{minted}[frame=single,fontsize=\small]{haskell}
>>> t1 = ("?a -> ?a" :: Type')
>>> t2 = ("?a -> ?b" :: Type')
>>> FreeFoil.alphaEquiv Foil.emptyScope t1 t2
True

>>> e1 = ("1" :: Exp')
>>> e2 = ("2" :: Exp')
>>> FreeFoil.alphaEquiv Foil.emptyScope e1 e2
True
  \end{minted}
  \caption[\texttt{FreeFoil.alphaEquiv} usage examples (1)]{Examples demonstrating that \texttt{FreeFoil.alphaEquiv} does not distinguish between different variable names or literal values.}
  \label{fig:freefoil-alphaequiv-literals}
\end{figure}

\begin{figure}[H]
  \begin{minted}[frame=single,fontsize=\small]{haskell}
>>> t1 = ("forall a. (forall b. a -> b)" :: Type')
>>> t2 = ("forall x. (forall y. x -> y)" :: Type')
>>> t3 = ("forall b. (forall a. a -> b)" :: Type')
>>> FreeFoil.alphaEquiv Foil.emptyScope t1 t2
True
>>> FreeFoil.alphaEquiv Foil.emptyScope t1 t3
False
  \end{minted}
  \caption[\texttt{FreeFoil.alphaEquiv} usage examples (2)]{Examples demonstrating that \texttt{FreeFoil.alphaEquiv} considers types with different variable names as equivalent, but is sensitive to the order of nested \texttt{forall}s.}
  \label{fig:freefoil-alphaequiv-foralls}
\end{figure}

Formally, we consider two polymorphic types equal (see Figure~\ref{fig:equiv-poly-rule}) if there exists a substitution of their bound type variables such that, after applying this substitution, the two types become syntactically identical, i.e. alpha-equivalent, regardless of the order in which the "forall" quantifiers appear. In the implementation, we compare two polymorphic types using the \texttt{equivPoly} function (see Figure~\ref{fig:equiv-poly-impl}), which works as follows.

First, we generalize both types for all present variables and immediately specialize them (lines 3 and 4). By doing this, we re-assign all existing free variables, if any, with the fresh ones, which is necessary to make sure all variables are present in the \texttt{tcLevelMap} in the \texttt{TypeInferencer} context, which is used later. Also, specialized types now can only have free type variables, i.e. no \texttt{forall}s will be present in the types. Moreover, \texttt{xs} and \texttt{ys} returned by \texttt{specialize} contain all type variables in the types, since they were generalized before, so at this step we may already conclude that \texttt{l} and \texttt{r} are not equal, if the number of unique type variables differs, that is \texttt{length xs /= length ys} (lines 5 and 6). If it is not the case, we proceed further.

Next, we actually need to perform the alpha equivalence check. However, as was mentioned earlier, \texttt{FreeFoil.alphaEquiv} will not work correctly with the free variables. We solve the problem by using the already mentioned unification algorithm in order to unify a single constraint, in other words, we solve a single equation of form $l = r$ to get a substitution \texttt{subst}.

Finally, after obtaining the substitution, we check that it matches all type variables from the left type to those in the right type, and that all variables are accounted for, ensuring a one-to-one correspondence between variables in both types. If the substitution is not found, or type variables in \texttt{l} do not correspond to type variables in \texttt{l}, we conclude that types are not equal. The reason why \texttt{equivPoly} works in the \texttt{TypeInferencer} context and uses the \texttt{tcLevelMap} is only because \texttt{specialize} and \texttt{unifyConstraint} need this, which, could be avoided, but kept for the sake of simplicity of the implementation.

\begin{figure}[H]
  \begin{prooftree*}
    \hypo{T_2 = [x_1 \mapsto y_{i_1}, \ldots, x_n \mapsto y_{i_n}] T_1}
    \hypo{\{y_{i_1}, \ldots, y_{i_n}\} = \{y_1, \ldots, y_n\}}
    \infer2{ \forall \{x_1, \ldots, x_n\}. T_1 = \forall \{y_1, \ldots, y_n\}. T_2}
  \end{prooftree*}
  \caption{Equivalence of polytypes}
  \label{fig:equiv-poly-rule}
\end{figure}

\begin{figure}[H]
  \begin{minted}[frame=single,fontsize=\small,linenos]{haskell}
equivPoly :: Type' -> Type' -> TypeInferencer n Bool
equivPoly l r = do
  (l', xs) <- specialize $ genAll l
  (r', ys) <- specialize $ genAll r
  if length xs /= length ys
    then return False
    else do
      levelMap <- gets tcLevelMap
      case unifyConstraint levelMap (Constraint (l', r')) of
        Left _ -> return False
        Right (Subst subst, _) -> do
          let matchings = [(x, y) | (x, TUVar y) <- Map.toList subst]
              allXs = List.sort xs == List.sort (map fst matchings)
              allYs = List.sort ys == List.sort (map snd matchings)
          return (allXs && allYs)
  where
    genAll t = generalizeWithIdents (Set.toList (freeVars t)) t
  \end{minted}
  \caption{\texttt{equivPoly} function for equivalence check of polytypes}
  \label{fig:equiv-poly-impl}
\end{figure}

\chapter{Evaluation and Discussion}
\label{chap:evaluation}

This chapter analyzes the prototype type-checker presented in Chapter~\ref{chap:implementation}. We evaluate the design choices, discuss the strengths and limitations of the implementation, and reflect on the practical experience of using BNFC and Free~Foil for prototyping. Finally, we propose steps for the future work.

\section{Results}

Using \emph{BNFC} to generate the front-end and \emph{Free~Foil} to generate scope-safe second-order abstract syntax, accelerated the type-checker prototype development. Together with GHC advanced features (e.g., rank-N polymorphism, GADTs, and Template~Haskell), scope-safety and generality of the abstract syntax generated with Free~Foil eliminated an entire class of bugs and allowed generic functions, such as substitution, to be written independently from the concrete language terms. The main drawback of Free~Foil observed is the learning curve — kind-level encodings of scopes and compiler error messages can be confusing.

All tests focused on algorithm correctness passed, giving empirical evidence of soundness. During testing we have also uncovered a problem related to comparison of polymorphic types, which could be further investigated to improve the Free~Foil framework.

We highlight the main advantages of the chosen approach:

\begin{description}
  \item[\textbf{Safety}] By outsourcing handling of bound names to Free~Foil, capture-avoiding substitution and operations with scoped AST are implemented in safe manner, and encoding of rapier's invariants on the type level, prevents the introduction of bugs.
  \item[\textbf{Modularity}] The type-checker is parameterized over the abstract syntax functor; adding records or algebraic data types would mainly require new pattern synonyms and no changes to the inference core.
  \item[\textbf{Clarity}] Haskell implementation follows directly the mathematical definition of the proposed algorithm $\mathcal{L}$ (Figure~\ref{fig:algorithm-L}).
\end{description}

\section{Limitations}

We also identify the following limitations of this work:

\begin{itemize}
  \item Unification variables are represented by \texttt{String}s and their levels stored in a \texttt{HashMap}. A more efficient design would be to integrate a level directly into the free-foil's binder data type, eliminating operations on map entirely.
  \item The Robinson-style unification algorithm is implemented "by hand", re-traversing terms for occurs checks and level map updating. Outsourcing to a well-tested generic unification library would both speed up the type-checker, and give us more confidence that the implementation is correct.
  \item As a proof-of-concept, the focus of this work is made on correctness and clarity; no benchmarks were performed. To better evaluate performance and scalability, more sophisticated test cases, either based on real-world projects or generated with scripts, should be added and benchmarked.
\end{itemize}

\section{Future Work}

Besides possible improvements of the proposed implementation listed in the previous section, we propose a possible way to generalize our approach to build a generic Hindley-Milner-style type checking library on top of Free~Foil.

A library would provide:

\begin{itemize}
  \item Type-class that a user should implement for their AST signature. Figure~\ref{fig:HMTypingSig} below shows a possible declaration of a such type-class — \texttt{HMTypingSig}.
  \item Type inference function that would take an AST node with the defined instance of \texttt{HMTypingSig}, and actually run type inference algorithm supplying the necessary initial context, and performing post-processing.
  \item Set of functions available to a user in the context of type inference. Utility-functions used in the \texttt{TypeInferencer} context (for example, \texttt{enterScope}, \texttt{addConstraints}) could be provided to a user as the "building blocks" for their type inference rules.
\end{itemize}

\begin{figure}[H]
\begin{minted}[frame=single,fontsize=\small]{haskell}
class HMTypingSig
  (binder :: Foil.S -> Foil.S -> *)
  (typeSig :: * -> * -> *)
  (sig :: * -> * -> *) 
  where
  inferType
    :: sig (ScopedInfer binder typeSig n) (Infer binder typeSig)
    -> TypeInferencer (UType binder typeSig) n (Infer binder typeSig)

instance HMTypingSig FoilTypePattern TypeSig ExpSig where
  inferType = \case
    ETrueSig -> do ...
\end{minted}
  \caption{Proposed algorithm generalization via \texttt{HMTypingSig} type-class.}
  \label{fig:HMTypingSig}
\end{figure}

In the current implementation, one major problem we see is the non-obvious call to \texttt{unify} when inferring the type of a let-binding. To solve this, we suggest to modify the proposed implementation adopting the algorithm introduced by Heeren \emph{et~al.}\cite{Heeren2002_GeneralizingHM}. Then, calls to \texttt{generalize}, \texttt{specialize}, and \texttt{unify} would be replaced by addition of two new constraint kinds, namely \emph{explicit instance constraint} ($\tau \preceq \sigma$) and \emph{implicit instance constraint} ($\tau_1 \leq_M \tau_2$), in addition to the only supported \emph{equality constraint} ($\tau_1 \equiv \tau_2$). As a result, a user would only need to think about \emph{what kind of constraint to add}, instead of \emph{when to call these methods}, when implementing the signature.

\chapter{Conclusion}
\label{chap:conclusion}

This work demonstrates that modern functional programming and code generation tools can be combined to build an extensible, scope-safe Hindley-Milner type-checker. Using BNFC, Free Foil, and Haskell, we developed a prototype with level-based generalization as a proof of concept.

Our main contributions are the following. First, we presented an algorithm for Hindley-Milner type inference with level-based generalization that separates constraint generation from solving — algorithm $\mathcal{L}$. Second, we implemented the algorithm in Haskell using generic, scope-safe abstract syntax generated through Free~Foil. Third, we developed a test-suite and validated our implementation against it, which, to some extent, empirically verifies the correctness of the algorithm. Finally, we have proposed practical next steps for evolving the prototype into a reusable type inference library.

The contributions of this work have also been presented as part of a talk at WITS'25\footnote{\textit{Towards Generic Type Checking Implementations in Haskell via Second-Order Abstract Syntax}, Workshop on the Implementation of Type Systems, \url{https://popl25.sigplan.org/details/wits-2025-papers/6/Towards-Generic-Type-Checking-Implementations-in-Haskell-via-Second-Order-Abstract-Sy}}.

\chapter{Acknowledgments}
\label{chap:ack}

I would like to thank my supervisor, Nikolai Kudasov, for his advice, and invaluable support in learning type theory and Haskell, as well as for his thorough approach to work and science in general.

I am also very grateful to my colleagues, Diana Tomilovskaia and Anastasia Smirnova, for their feedback and insights, which we shared during the discussions related to our works.

\printbibliography[heading=bibintoc,title={References}]

\end{document}
