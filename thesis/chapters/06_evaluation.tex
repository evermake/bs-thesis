\chapter{Evaluation and Discussion}
\label{chap:evaluation}

%%%%%%%%%%%%%%%%%%%%%%%%%%%%%%%%%%%%%%%%%%%%%%%%%%%%%%%%%%%%%%%%%%%%%%%%%%%%%%%%
% подробно расписываю анализ своей реализации, что получилось хорошо, что нет, на какие компромиссы пришлось пойти + вывод “как можно использовать то, что я сделал в этой работе“
% тут же Future Work:
%  - Обобщение всего этого подхода. Сейчас “заложен фундамент”, в дальнейшем можно сделать обобщённый Hindley-Milner с уровнями.
%  - Использование эффективных структур данных и алгоритма унификации через существующие библиотеки.
%  - Моя работа это Proof-of-Concept: есть Free Foil, в него можно добавить уровни

% Тут должен быть ответ на вопрос: “что всё это значит?”, выводы из моей работы.
% Возможные выводы:
%   Haskell хорош/плох для таких-то задач
%   free-foil хорош/плох для реализации тайпчекера (безопасность, скорость и т.п.) или наоборот какие-то трудности
%   сравнение производительности
%%%%%%%%%%%%%%%%%%%%%%%%%%%%%%%%%%%%%%%%%%%%%%%%%%%%%%%%%%%%%%%%%%%%%%%%%%%%%%%%
