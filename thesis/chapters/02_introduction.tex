\chapter{Introduction}
\label{chap:intro}

%%%%%%%%%%%%%%%%%%%%%%%%%%%%%%%%%%%%%%%%%%%%%%%%%%%%%%%%%%%%%%%%%%%%%%%%%%%%%%%%
% - Explain the (larger) context of your work (with references!)
% - Give a simple, but descriptive example of a problem that you are solving, or a problem that is a direct application of your work. The example should be maximally self-contained and be simple enough to convey the basic idea to the reader.
% - Provide an brief overview of approaches to the problem (with references!)
% - Briefly explain what are the ideaa behind your work and how it compares against other approaches.
% - Clearly state the contributions of your work.
%%%%%%%%%%%%%%%%%%%%%%%%%%%%%%%%%%%%%%%%%%%%%%%%%%%%%%%%%%%%%%%%%%%%%%%%%%%%%%%%

Type system is a crucial part of any formal system, for example a programming language or a proof assistant.

When implementing a programming language with type safety one should implement:
\begin{enumerate}
  \item terms with bound identifiers — we want to support variables
  \item polymorphism — we want to support generic algorithms
  \item extendability — we want to easily add new features to language
  \item performance — we want compiler/typechecker to be fast
\end{enumerate}

...

Free Foil of Kudasov \textit{et al.} \cite{FreeFoil} has shown that it is possible to outsource the implementation of bound identifiers and scopes by generating a scope-safe abstract syntax in Haskell, thus allowing a user of their framework to add support of binders to their language with extremely small effort.

...

Definitely, numerous implementations of the Hindley-Milner type system can be found in both academic literature and open-source projects. what's different in this work is that we use a novel framework (Free Foil) based on efficient Rapier (used in GHC) for working with name binders, and not widely adopted idea of Levels (used in OCaml).

...
