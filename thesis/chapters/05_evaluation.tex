\chapter{Evaluation and Discussion}
\label{chap:evaluation}

This chapter analyzes the prototype type-checker presented in Chapter~\ref{chap:implementation}. We evaluate the design choices, discuss the strengths and limitations of the implementation, and reflect on the practical experience of using BNFC and Free~Foil for prototyping. Finally, we propose steps for the future work.

\section{Results}

Using \emph{BNFC} to generate the front-end and \emph{Free~Foil} to generate scope-safe second-order abstract syntax, accelerated the type-checker prototype development. Together with GHC advanced features (e.g., rank-N polymorphism, GADTs, and Template~Haskell), scope-safety and generality of the abstract syntax generated with Free~Foil eliminated an entire class of bugs and allowed generic functions, such as substitution, to be written independently from the concrete language terms. The main drawback of Free~Foil observed is the learning curve — kind-level encodings of scopes and compiler error messages can be confusing.

All tests focused on algorithm correctness passed, giving empirical evidence of soundness. During testing we have also uncovered a problem related to comparison of polymorphic types, which could be further investigated to improve the Free~Foil framework.

We highlight the main advantages of the chosen approach:

\begin{description}
  \item[\textbf{Safety}] By outsourcing handling of bound names to Free~Foil, capture-avoiding substitution and operations with scoped AST are implemented in safe manner, and encoding of rapier's invariants on the type level, prevents the introduction of bugs.
  \item[\textbf{Modularity}] The type-checker is parameterized over the abstract syntax functor; adding records or algebraic data types would mainly require new pattern synonyms and no changes to the inference core.
  \item[\textbf{Clarity}] Haskell implementation follows directly the mathematical definition of the proposed algorithm $\mathcal{L}$ (Figure~\ref{fig:algorithm-L}).
\end{description}

\section{Limitations}

We also identify the following limitations of this work:

\begin{itemize}
  \item Unification variables are represented by \texttt{String}s and their levels stored in a \texttt{HashMap}. A more efficient design would be to integrate a level directly into the free-foil's binder data type, eliminating operations on map entirely.
  \item The Robinson-style unification algorithm is implemented "by hand", re-traversing terms for occurs checks and level map updating. Outsourcing to a well-tested generic unification library would both speed up the type-checker, and give us more confidence that the implementation is correct.
  \item As a proof-of-concept, the focus of this work is made on correctness and clarity; no benchmarks were performed. To better evaluate performance and scalability, more sophisticated test cases, either based on real-world projects or generated with scripts, should be added and benchmarked.
\end{itemize}

\section{Future Work}

Besides possible improvements of the proposed implementation listed in the previous section, we propose a possible way to generalize our approach to build a generic Hindley-Milner-style type checking library on top of Free~Foil.

A library would provide:

\begin{itemize}
  \item Type-class that a user should implement for their AST signature. Figure~\ref{fig:HMTypingSig} below shows a possible declaration of a such type-class — \texttt{HMTypingSig}.
  \item Type inference function that would take an AST node with the defined instance of \texttt{HMTypingSig}, and actually run type inference algorithm supplying the necessary initial context, and performing post-processing.
  \item Set of functions available to a user in the context of type inference. Utility-functions used in the \texttt{TypeInferencer} context (for example, \texttt{enterScope}, \texttt{addConstraints}) could be provided to a user as the "building blocks" for their type inference rules.
\end{itemize}

\begin{figure}[H]
\begin{minted}[frame=single,fontsize=\small]{haskell}
class HMTypingSig
  (binder :: Foil.S -> Foil.S -> *)
  (typeSig :: * -> * -> *)
  (sig :: * -> * -> *) 
  where
  inferType
    :: sig (ScopedInfer binder typeSig n) (Infer binder typeSig)
    -> TypeInferencer (UType binder typeSig) n (Infer binder typeSig)

instance HMTypingSig FoilTypePattern TypeSig ExpSig where
  inferType = \case
    ETrueSig -> do ...
\end{minted}
  \caption{Proposed algorithm generalization via \texttt{HMTypingSig} type-class.}
  \label{fig:HMTypingSig}
\end{figure}

In the current implementation, one major problem we see is the non-obvious call to \texttt{unify} when inferring the type of a let-binding. To solve this, we suggest to modify the proposed implementation adopting the algorithm introduced by Heeren \emph{et~al.}\cite{Heeren2002_GeneralizingHM}. Then, calls to \texttt{generalize}, \texttt{specialize}, and \texttt{unify} would be replaced by addition of two new constraint kinds, namely \emph{explicit instance constraint} ($\tau \preceq \sigma$) and \emph{implicit instance constraint} ($\tau_1 \leq_M \tau_2$), in addition to the only supported \emph{equality constraint} ($\tau_1 \equiv \tau_2$). As a result, a user would only need to think about \emph{what kind of constraint to add}, instead of \emph{when to call these methods}, when implementing the signature.
