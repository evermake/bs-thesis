\chapter{\todored{Evaluation and Discussion}}
\label{chap:evaluation}

%%%%%%%%%%%%%%%%%%%%%%%%%%%%%%%%%%%%%%%%%%%%%%%%%%%%%%%%%%%%%%%%%%%%%%%%%%%%%%%%
% подробно расписываю анализ своей реализации, что получилось хорошо, что нет, на какие компромиссы пришлось пойти + вывод “как можно использовать то, что я сделал в этой работе“
% тут же Future Work:
%  - Обобщение всего этого подхода. Сейчас “заложен фундамент”, в дальнейшем можно сделать обобщённый Hindley-Milner с уровнями.
%  - Использование эффективных структур данных и алгоритма унификации через существующие библиотеки.
%  - Моя работа это Proof-of-Concept: есть Free Foil, в него можно добавить уровни

% Тут должен быть ответ на вопрос: “что всё это значит?”, выводы из моей работы.
% Возможные выводы:
%   Haskell хорош/плох для таких-то задач
%   free-foil хорош/плох для реализации тайпчекера (безопасность, скорость и т.п.) или наоборот какие-то трудности
%   сравнение производительности
%%%%%%%%%%%%%%%%%%%%%%%%%%%%%%%%%%%%%%%%%%%%%%%%%%%%%%%%%%%%%%%%%%%%%%%%%%%%%%%%

\section{Conclusions}

\begin{enumerate}
  \item Using free foil and BNFC makes its' easy to add new terms: we only need to extend grammar, and: (1) parser and abstract syntax will be automatically generated with BNFC and Free Foil (2) places that are non-generic will be highlighted by Haskell's type system (3) generic algorithms will not even require any modifications, since they are generic
\end{enumerate}

\section{Future Work}

\begin{itemize}
  \item it's a toy language, possible problems may arise when implementing features for the real language (error messages, code editing suggestions)
  \item unification is crucial; algorithm presented here is simple and not proved theoretically, we could also outsource it too to other libraries
\end{itemize}
