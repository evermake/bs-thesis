\chapter{Введение}
\label{chap:intro}

Современные языки функционального программирования активно используют вывод типов без явных аннотаций программиста. Классическая система Хиндли–Милнера (HM) и алгоритм $\mathcal{W}$~\cite{Milner1978_TypePolymorphism} демонстрируют высокую популярность благодаря доказанной полноте и простоте реализации. Однако при расширении системы типов или увеличении размера проектов стандартный обход всей среды типов для let-обобщения становится узким местом. В работе предложено решение на основе уровневого обобщения, где каждая переменная унификации получает метку уровня, соответствующую глубине вложенности let-выражений~\cite{Remy1992_SortedEqTheoryTypes}. Благодаря этому обобщаются лишь переменные с уровнем, большим текущего, без необходимости обходить окружение.

Дополнительно, работа акцентирует внимание на проблемах управления связанными переменными в абстрактных синтаксических деревьях: избыточная работа по избежанию захвата переменных и сложность расширения базового алгебраического синтаксиса. Для решения этих задач используется фреймворк Free~Foil~\cite{FreeFoil}, объединяющий преимущества таких подходов как foil~\cite{Foil} и data types à la carte \cite{Swierstra2008_a_la_carte}. Free Foil генерирует обобщённое безопасное для областей видимости представление АСД, минимизируя ручную работу при поддержке связанных переменных.

Цель исследования: спроектировать, реализовать и проверить прототип алгоритма вывода типов HM с уровневым обобщением и разделёнными фазами сбора и решения ограничений, применимый к языку с лямбда-выражениями, натуральными и булевыми литералами, let-полиморфизмом, а также предложить дальнейшие шаги для развития прототипа в библиотеку для вывода типов.
