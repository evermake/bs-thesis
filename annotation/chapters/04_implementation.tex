\chapter{Реализация}
\label{chap:impl}

Прототип реализован на языке программирования Haskell с использованием инструментов BNFC, Alex, Happy и Template~Haskell\footnote{Template Haskell — расширение GHC, позволяющее выполнять метапрограммирование, то есть генерировать код во время компиляции.} для генерации АСД. На основе грамматики языка в нотации LBNF, с помощью BNFC генерируются необходимые модули с парсером языка и определением АСД первого порядка. С помощью Free~Foil генерируется обобщённый абстрактный синтаксис второго порядка для сигнатур выражения \texttt{ExpSig} и \texttt{TypeSig}.

Реализована монада \texttt{TypeInferencer} для управления контекстом алгоритма типизации с необходимыми функциями для выполнения операций в контексте монады. Основываясь на упомянутых утилитах и монаде, функция \texttt{inferType} рекурсивно обрабатывает узлы АСД и вызывает необходимые операции, возвращая выведенный тип программы. Реализация функции вывода типа, соответствующая алгоритму $\mathcal{L}$, представлена на рисунке~\ref{fig:inferType}.

\begin{figure}[H]
  \begin{minted}[frame=single,fontsize=\small]{haskell}
inferType :: Exp n -> TypeInferencer n Type'
inferType ETrue = return TBool
inferType EFalse = return TBool
inferType (ENat _) = return TNat
inferType (FreeFoil.Var x) = do
  TypingEnv env <- gets tcEnv
  specialize_ (Foil.lookupName x env)  
inferType (EAbs (FoilPatternVar x) eBody) = do
  tParam <- freshUVar_
  tBody <- enterScope x tParam $ inferType eBody
  return (TArrow tParam tBody)
inferType (EApp eAbs eArg) = do
  tAbs <- inferType eAbs
  tArg <- inferType eArg
  tRes <- freshUVar_
  addConstraints [(tAbs, TArrow tArg tRes)]
  return tRes
inferType (ELet eBound (FoilPatternVar x) eInner) = do
  tBound <- enterLevel (inferType eBound)
  unify
  subst <- gets tcSubst
  tBound' <- generalize (applySubst subst tBound)
  enterScope x tBound' (inferType eInner)
\end{minted}
  \caption[Реализация \texttt{inferType}]{Реализация функции \texttt{inferType}.}
  \label{fig:inferType}
\end{figure}

В работе также описан подход к тестированию и разработан набор тестов, состоящий из 33 корректных и 19 некорректных программ. Все тесты выполнены успешно, что эмпирически подтверждает корректность реализации.
